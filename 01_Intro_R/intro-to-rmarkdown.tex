% Options for packages loaded elsewhere
\PassOptionsToPackage{unicode}{hyperref}
\PassOptionsToPackage{hyphens}{url}
%
\documentclass[
]{article}
\usepackage{amsmath,amssymb}
\usepackage{lmodern}
\usepackage{ifxetex,ifluatex}
\ifnum 0\ifxetex 1\fi\ifluatex 1\fi=0 % if pdftex
  \usepackage[T1]{fontenc}
  \usepackage[utf8]{inputenc}
  \usepackage{textcomp} % provide euro and other symbols
\else % if luatex or xetex
  \usepackage{unicode-math}
  \defaultfontfeatures{Scale=MatchLowercase}
  \defaultfontfeatures[\rmfamily]{Ligatures=TeX,Scale=1}
\fi
% Use upquote if available, for straight quotes in verbatim environments
\IfFileExists{upquote.sty}{\usepackage{upquote}}{}
\IfFileExists{microtype.sty}{% use microtype if available
  \usepackage[]{microtype}
  \UseMicrotypeSet[protrusion]{basicmath} % disable protrusion for tt fonts
}{}
\makeatletter
\@ifundefined{KOMAClassName}{% if non-KOMA class
  \IfFileExists{parskip.sty}{%
    \usepackage{parskip}
  }{% else
    \setlength{\parindent}{0pt}
    \setlength{\parskip}{6pt plus 2pt minus 1pt}}
}{% if KOMA class
  \KOMAoptions{parskip=half}}
\makeatother
\usepackage{xcolor}
\IfFileExists{xurl.sty}{\usepackage{xurl}}{} % add URL line breaks if available
\IfFileExists{bookmark.sty}{\usepackage{bookmark}}{\usepackage{hyperref}}
\hypersetup{
  pdftitle={Introduction to R Markdown},
  hidelinks,
  pdfcreator={LaTeX via pandoc}}
\urlstyle{same} % disable monospaced font for URLs
\usepackage[margin=1in]{geometry}
\usepackage{color}
\usepackage{fancyvrb}
\newcommand{\VerbBar}{|}
\newcommand{\VERB}{\Verb[commandchars=\\\{\}]}
\DefineVerbatimEnvironment{Highlighting}{Verbatim}{commandchars=\\\{\}}
% Add ',fontsize=\small' for more characters per line
\usepackage{framed}
\definecolor{shadecolor}{RGB}{248,248,248}
\newenvironment{Shaded}{\begin{snugshade}}{\end{snugshade}}
\newcommand{\AlertTok}[1]{\textcolor[rgb]{0.94,0.16,0.16}{#1}}
\newcommand{\AnnotationTok}[1]{\textcolor[rgb]{0.56,0.35,0.01}{\textbf{\textit{#1}}}}
\newcommand{\AttributeTok}[1]{\textcolor[rgb]{0.77,0.63,0.00}{#1}}
\newcommand{\BaseNTok}[1]{\textcolor[rgb]{0.00,0.00,0.81}{#1}}
\newcommand{\BuiltInTok}[1]{#1}
\newcommand{\CharTok}[1]{\textcolor[rgb]{0.31,0.60,0.02}{#1}}
\newcommand{\CommentTok}[1]{\textcolor[rgb]{0.56,0.35,0.01}{\textit{#1}}}
\newcommand{\CommentVarTok}[1]{\textcolor[rgb]{0.56,0.35,0.01}{\textbf{\textit{#1}}}}
\newcommand{\ConstantTok}[1]{\textcolor[rgb]{0.00,0.00,0.00}{#1}}
\newcommand{\ControlFlowTok}[1]{\textcolor[rgb]{0.13,0.29,0.53}{\textbf{#1}}}
\newcommand{\DataTypeTok}[1]{\textcolor[rgb]{0.13,0.29,0.53}{#1}}
\newcommand{\DecValTok}[1]{\textcolor[rgb]{0.00,0.00,0.81}{#1}}
\newcommand{\DocumentationTok}[1]{\textcolor[rgb]{0.56,0.35,0.01}{\textbf{\textit{#1}}}}
\newcommand{\ErrorTok}[1]{\textcolor[rgb]{0.64,0.00,0.00}{\textbf{#1}}}
\newcommand{\ExtensionTok}[1]{#1}
\newcommand{\FloatTok}[1]{\textcolor[rgb]{0.00,0.00,0.81}{#1}}
\newcommand{\FunctionTok}[1]{\textcolor[rgb]{0.00,0.00,0.00}{#1}}
\newcommand{\ImportTok}[1]{#1}
\newcommand{\InformationTok}[1]{\textcolor[rgb]{0.56,0.35,0.01}{\textbf{\textit{#1}}}}
\newcommand{\KeywordTok}[1]{\textcolor[rgb]{0.13,0.29,0.53}{\textbf{#1}}}
\newcommand{\NormalTok}[1]{#1}
\newcommand{\OperatorTok}[1]{\textcolor[rgb]{0.81,0.36,0.00}{\textbf{#1}}}
\newcommand{\OtherTok}[1]{\textcolor[rgb]{0.56,0.35,0.01}{#1}}
\newcommand{\PreprocessorTok}[1]{\textcolor[rgb]{0.56,0.35,0.01}{\textit{#1}}}
\newcommand{\RegionMarkerTok}[1]{#1}
\newcommand{\SpecialCharTok}[1]{\textcolor[rgb]{0.00,0.00,0.00}{#1}}
\newcommand{\SpecialStringTok}[1]{\textcolor[rgb]{0.31,0.60,0.02}{#1}}
\newcommand{\StringTok}[1]{\textcolor[rgb]{0.31,0.60,0.02}{#1}}
\newcommand{\VariableTok}[1]{\textcolor[rgb]{0.00,0.00,0.00}{#1}}
\newcommand{\VerbatimStringTok}[1]{\textcolor[rgb]{0.31,0.60,0.02}{#1}}
\newcommand{\WarningTok}[1]{\textcolor[rgb]{0.56,0.35,0.01}{\textbf{\textit{#1}}}}
\usepackage{graphicx}
\makeatletter
\def\maxwidth{\ifdim\Gin@nat@width>\linewidth\linewidth\else\Gin@nat@width\fi}
\def\maxheight{\ifdim\Gin@nat@height>\textheight\textheight\else\Gin@nat@height\fi}
\makeatother
% Scale images if necessary, so that they will not overflow the page
% margins by default, and it is still possible to overwrite the defaults
% using explicit options in \includegraphics[width, height, ...]{}
\setkeys{Gin}{width=\maxwidth,height=\maxheight,keepaspectratio}
% Set default figure placement to htbp
\makeatletter
\def\fps@figure{htbp}
\makeatother
\setlength{\emergencystretch}{3em} % prevent overfull lines
\providecommand{\tightlist}{%
  \setlength{\itemsep}{0pt}\setlength{\parskip}{0pt}}
\setcounter{secnumdepth}{-\maxdimen} % remove section numbering
\ifluatex
  \usepackage{selnolig}  % disable illegal ligatures
\fi

\title{Introduction to R Markdown}
\author{}
\date{\vspace{-2.5em}}

\begin{document}
\maketitle

You will be submitting your final project in the form of an R Markdown
file (.Rmd) and a knitted HTML file (.html). This tutorial will help you
navigate the R Markdown platform.

\hypertarget{markdown}{%
\subsection{Markdown}\label{markdown}}

Markdown is a simplified version of `markup' languages. Unlike
cumbersome word processing applications, text written in Markdown uses
simple and intuitve formatting elements and can be easily shared between
computers, mobile phones, and people. It's quickly becoming the writing
standard for academics, scientists, writers, and many more. Websites
like GitHub and reddit use Markdown to style their comments.

Formatting text in Markdown has a very gentle learning curve. It doesn't
do anything fancy like change the font size, color, or type - just the
essentials, using keyboard symbols you already know. All you have
control over is the display of the text-stuff like making things bold,
creating headers, and organizing lists.

We won't go into a lot of detail, but run through a few basic examples.
For more details and examples, click
\href{https://www.markdowntutorial.com/}{here}.

\hypertarget{italics-and-bold}{%
\subsubsection{Italics and Bold}\label{italics-and-bold}}

To make a phrase italic in Markdown, you can surround words with a
single underscore (\texttt{\_}) or asterisk (\texttt{*}). For example,
\emph{this} word would become italic.

Similarly, to make phrases bold in Markdown, you can surround words with
two underscores (\texttt{\_\_}) or two asterisks ( \texttt{**} ). This
will \textbf{really} get your point across.

Most commonly, a single underscore is used for italics and two asterisks
are used for bold.

Of course, you can use \textbf{\emph{both}} italics and bold in the same
line by adding both an underscore and asterisks (\texttt{**\_}). You can
also span them \textbf{\emph{across multiple words}}.

In general, it doesn't matter which order you place the asterisks or
underscores. I prefer to place the asterisks on the outside to make it
\textbf{\emph{easier to read}}.

\hypertarget{headers}{%
\subsubsection{Headers}\label{headers}}

Headers are frequently used on websites, magazine articles, and notices,
to draw attention to a section. As their name implies, they act like
titles or subtitles above sections.

There are six types of headers, in decreasing sizes:

\hypertarget{header-one}{%
\section{Header one}\label{header-one}}

\hypertarget{header-two}{%
\subsection{Header two}\label{header-two}}

\hypertarget{header-three}{%
\subsubsection{Header three}\label{header-three}}

\hypertarget{header-four}{%
\paragraph{Header four}\label{header-four}}

\hypertarget{header-five}{%
\subparagraph{Header five}\label{header-five}}

Header six

To make headers in Markdown, you preface the phrase with a hash mark
(\texttt{\#}). You place the same number of hash marks as the size of
the header you want. For example, for a header one, you'd use one hash
mark (\texttt{\#\ Header\ One}), while for a header three, you'd use
three (\texttt{\#\#\#\ Header\ Three}). It's up to you to decide when
it's appropriate to use which header. In general, headers one and six
should be used sparingly.

\hypertarget{links}{%
\subsubsection{Links}\label{links}}

There are two different link types in Markdown, but both of them render
the exact same way. The first link style is called an \emph{inline}
link. To create an inline link, you wrap the link text in brackets
(\texttt{{[}\ {]}}), and then you wrap the link in parenthesis
(\texttt{(\ )}). For example, to create a hyperlink to www.github.com,
with a link text that says, Visit GitHub!, you'd write this in Markdown:
\href{https://github.com/}{Visit GitHub}.

The other link type is called a reference link. As the name implies, the
link is actually a reference to another place in the document. Here's an
example:

Here's \href{https://github.com/}{a link to something else}.\\
Here's \href{https://www.google.com/}{yet another link}.\\
And now back to \href{https://github.com/}{the first link}.

The ``references'' above are the second set of brackets:
\href{https://github.com/}{another place} and
\href{https://www.google.com/}{another link}. At the bottom of a
Markdown document, these brackets are defined as proper links to outside
websites. An advantage of the reference link style is that multiple
links to the same place only need to be updated once. For example, if we
decide to make all of the \href{https://github.com/}{another place}
links go somewhere else, we only have to change the single reference
link.

Reference links don't appear in the rendered Markdown. You define them
by providing the same tag name wrapped in brackets, followed by a colon,
followed by the link.

\hypertarget{images}{%
\subsubsection{Images}\label{images}}

If you know how to create links in Markdown, you can create images, too.
The syntax is nearly the same.

Images also have two styles, just like links, and both of them render
the exact same way. The difference between links and images is that
images are prefaced with an exclamation point (\texttt{!}).

The first image style is called an \emph{inline image link}. To create
an inline image link, enter an exclamation point (\texttt{!}), wrap the
alt text in brackets (\texttt{{[}\ {]}}), and then wrap the link in
parenthesis (\texttt{(\ )}). (Alt text is a phrase or sentence that
describes the image for the visually impaired.)

For example, to create an inline image link to
\url{https://octodex.github.com/images/bannekat.png}, with an alt text
that says, Benjamin Bannekat, you'd write this in Markdown:

\begin{figure}
\centering
\includegraphics[width=0.5\textwidth,height=\textheight]{https://octodex.github.com/images/bannekat.png}
\caption{Benjamin Bannekat}
\end{figure}

Although you don't \emph{need} to add alt text, it will make your
content accessible to your audience, including people who are visually
impaired, use screen readers, or do not have high speed internet
connections. The image I chose is actually much larger so I added
\texttt{\{\ width=50\%\ \}} to shrink it a bit.

For a reference image, you'll follow the same pattern as a reference
link. You'll precede the Markdown with an exclamation point, then
provide two brackets for the alt text, and then two more for the image
tag. At the bottom of your Markdown page, you'll define an image for the
tag.

Here's an example: the first reference tag is called ``First Father'',
and links to \url{http://octodex.github.com/images/founding-father.jpg};
the second image links out to
\url{http://octodex.github.com/images/foundingfather_v2.png}.

\begin{figure}
\centering
\includegraphics[width=0.5\textwidth,height=\textheight]{http://octodex.github.com/images/founding-father.jpg}
\caption{The first father}
\end{figure}

\begin{figure}
\centering
\includegraphics[width=0.5\textwidth,height=\textheight]{http://octodex.github.com/images/foundingfather_v2.png}
\caption{The second first father}
\end{figure}

\hypertarget{lists}{%
\subsubsection{Lists}\label{lists}}

There are two types of lists in the known universe: unordered and
ordered. That's a fancy way of saying that there are lists with bullet
points, and lists with numbers.

To create an unordered list, you'll want to preface each item in the
list with an asterisk (\texttt{*}). Each list item also gets its own
line. For example, a grocery list in Markdown might look like this:

\begin{itemize}
\tightlist
\item
  Milk
\item
  Eggs
\item
  Salmon
\item
  Butter
\end{itemize}

An ordered list is prefaced with numbers, instead of asterisks. For
example:

\begin{enumerate}
\def\labelenumi{\arabic{enumi}.}
\tightlist
\item
  Crack three eggs over a bowl
\item
  Pour a gallon of milk into the bowl
\item
  Rub the salmon vigorously with butter
\item
  Drop the salmon into the egg-milk bowl
\end{enumerate}

Easy, right? It's just like you'd expect a list to look.

\hypertarget{knitr}{%
\subsection{knitr}\label{knitr}}

\texttt{knitr} is an R package that is used for \emph{statistical
literate programming}, meaning you are able to integrate code and text
in a single, simple document format. It supports \texttt{R\ Markdown},
\texttt{R\ LaTex}, and \texttt{R\ HTML} as documentation languages, and
can export \texttt{markdown}, \texttt{PDF} and \texttt{HTML} documents.

Something that is really nice about using \texttt{R\ Markdown} with
\texttt{knitr} is that your final document will not be created if there
is an error in your code. Thus, it's really easy to check if your code
is running by simply \texttt{knitting} the \texttt{.Rmd} file.

Make sure to install the latest version of the \texttt{knitr} package:
\texttt{install.packages("knitr")}.

\hypertarget{rmarkdown}{%
\subsection{rmarkdown}\label{rmarkdown}}

Documents, like this one, containing both R code (below) and markdown
are \texttt{R\ Markdown} files. The similarly named \texttt{rmarkdown}
is an R package that makes working with \texttt{R\ Markdown} easier by
wrapping \texttt{knitr} along with other tools. For more details on both
the \texttt{R\ Markdown} format and the \texttt{rmarkdown} package, see
\url{http://rmarkdown.rstudio.com}.

When you run \texttt{render} from the \texttt{rmarkdown} package (or
click the \textbf{Knit} button above), the \texttt{.Rmd} file is fed to
\texttt{knitr}, which executes all of the code chunks and creates a new
markdown (\texttt{.md}) document which includes the code and it's
output.

The markdown file generated by \texttt{knitr} is then processed by
pandoc which is responsible for creating the finished format.

\includegraphics{http://rmarkdown.rstudio.com/lesson-images/RMarkdownFlow.png}

\hypertarget{embedding-code}{%
\subsubsection{Embedding Code}\label{embedding-code}}

When you click the \textbf{Knit} button a document will be generated
that includes both content as well as the output of any embedded R `code
chunks' within the document. We \textbf{ALWAYS} want to see your code
and output. If you do not provide the code and output in the
\texttt{.Rmd} file, you will lose points. In addition to the
\texttt{.Rmd} file, we expect you to provide a knitted \texttt{.html}
file in your GitHub homework repository. We will also knit your
\texttt{.Rmd} file on our own computers while grading, to make sure we
can reproduce your results.

\hypertarget{code-chunks}{%
\paragraph{Code Chunks}\label{code-chunks}}

There are 3 ways to create a code chunk:

\begin{enumerate}
\def\labelenumi{\arabic{enumi}.}
\tightlist
\item
  Typing
  \texttt{\textasciigrave{}\textasciigrave{}\textasciigrave{}\{r\}} to
  start a code chunk and then
  \texttt{\textasciigrave{}\textasciigrave{}\textasciigrave{}} to end a
  code chunk\\
\item
  The keyboard shortcut \textbf{Ctrl + Alt + I} (OS X: \textbf{Cmd +
  Option + I})\\
\item
  Clicking the Add Chunk button in the tool bar
\end{enumerate}

Let's start with the built-in data set \texttt{pressure}, which includes
data on the vapor pressure of Mercury as a function of temperature. You
can embed an R code chunk like this:

\begin{Shaded}
\begin{Highlighting}[]
\NormalTok{x }\OtherTok{=} \DecValTok{2}
\end{Highlighting}
\end{Shaded}

\begin{Shaded}
\begin{Highlighting}[]
\FunctionTok{head}\NormalTok{(pressure)}
\end{Highlighting}
\end{Shaded}

\begin{verbatim}
##   temperature pressure
## 1           0   0.0002
## 2          20   0.0012
## 3          40   0.0060
## 4          60   0.0300
## 5          80   0.0900
## 6         100   0.2700
\end{verbatim}

\begin{Shaded}
\begin{Highlighting}[]
\FunctionTok{summary}\NormalTok{(pressure)}
\end{Highlighting}
\end{Shaded}

\begin{verbatim}
##   temperature     pressure       
##  Min.   :  0   Min.   :  0.0002  
##  1st Qu.: 90   1st Qu.:  0.1800  
##  Median :180   Median :  8.8000  
##  Mean   :180   Mean   :124.3367  
##  3rd Qu.:270   3rd Qu.:126.5000  
##  Max.   :360   Max.   :806.0000
\end{verbatim}

Note that both the code written and output produced are shown in the
final document.

\hypertarget{plots}{%
\subsubsection{Plots}\label{plots}}

\textbf{Assessment}: create a code chunk that makes a scatterplot of
temperature and pressure. When you are done, knit your file and look at
the ouput.

\hypertarget{inline-text-computations}{%
\subsubsection{Inline Text
Computations}\label{inline-text-computations}}

You can also include the output of your code in the middle of a
sentence. For example, I want to randomly generate two numbers, x and y,
and include their values in a sentence. I would first write the
following code chunk:

\begin{Shaded}
\begin{Highlighting}[]
\NormalTok{x }\OtherTok{\textless{}{-}} \FunctionTok{rnorm}\NormalTok{(}\DecValTok{1}\NormalTok{)}
\NormalTok{y }\OtherTok{\textless{}{-}} \FunctionTok{rnorm}\NormalTok{(}\DecValTok{1}\NormalTok{)}
\end{Highlighting}
\end{Shaded}

Then I can write x = 0.3161961 and y = -0.9558547.

\hypertarget{equations}{%
\subsubsection{Equations}\label{equations}}

If you know latex, including equations is really simple. The same syntax
is used. For example, you can write an \emph{inline} equation like this
- \(A = \pi*r^{2}\). You can also center an equation like this:

\begin{equation}
\mathbb{E}[Y] = \beta_0 + \beta_1x
\end{equation}

\hypertarget{the-cache-option}{%
\subsubsection{\texorpdfstring{The \texttt{cache}
option}{The cache option}}\label{the-cache-option}}

All code chunks have to be re-computed every time you re-knit the file.
If you have code chunks that take a while to process, you may want to
use the \texttt{cache\ =\ TRUE} option which stores and then loads the
results from cache after the first run, and can save you considerable
time. This can be done on a chunk-by-chunk basis. However, this is only
useful if you have code chunks you haven't edited since the first run.
If the data or code changes, you will have to re-run the code to update
the results

\hypertarget{extracting-r-code}{%
\subsubsection{Extracting R code}\label{extracting-r-code}}

In \texttt{knitr}, you can use \texttt{purl()} to pull out all of the R
code and put it into a single \texttt{.R} file. This wiil ignore all
prose outside of code chunks. The following code will create a file
called \texttt{intro-to-rmarkdown.R} in the same directory I'm working
in.

\begin{Shaded}
\begin{Highlighting}[]
\FunctionTok{library}\NormalTok{(knitr)}
\FunctionTok{purl}\NormalTok{(}\StringTok{"intro{-}to{-}rmarkdown.Rmd"}\NormalTok{, }\AttributeTok{documentation =} \DecValTok{0}\NormalTok{)}
\end{Highlighting}
\end{Shaded}

\hypertarget{summary}{%
\subsubsection{Summary}\label{summary}}

\includegraphics{https://sachsmc.github.io/knit-git-markr-guide/knitr/img/knitr-workflow.png}

\hypertarget{other-awesome-powers}{%
\subsubsection{Other Awesome Powers}\label{other-awesome-powers}}

R Markdown can render PDF presentations with beamer, HTML presentations
with ioslides, slidy and reveal.js. It can also be used to write full
academic manuscripts. You can also build your own websites and
interactive documents. These powers are too complicated and time
consuming for this lecture, but at least you know they're possible!

\end{document}
